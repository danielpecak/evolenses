\documentclass[11pt,a4paper,twocolumn]{article}
\usepackage[utf8]{inputenc}
\usepackage{polski}
\usepackage[polish]{babel}
\usepackage{tgtermes}
\usepackage{amsmath}
\usepackage{amsfonts}
%\usepackage{amssymb}
\usepackage{graphicx}
\usepackage[left=2cm,right=2cm,top=2cm,bottom=2cm]{geometry}
\author{Gronkiewicz, D., Pęcak, D.}
\title{Wzory do liczenia soczewek}

\renewcommand{\vector}[2]{\left(
\begin{array}{c}
 #1  \\
 #2  \\
\end{array}
\right)}

\newcommand{\ABCDmat}[4]{\left(
\begin{array}{cc}
 #1 & #2  \\
 #3 & #4  \\
\end{array}
\right)}

\begin{document}
\section{Notacja i konwencje}
Oś optyczna (oznaczona literą $z$) pozioma, propagacja z lewej do prawej.
\begin{itemize}
\item $R$ -- promień krzywizny granicy ośrodków. $R>0$ odpowiada wypukłości w lewą stronę.
\item $x$,$y$ -- współrzędne w płaszczyźnie optyki. W przypadku 1D używamy współrzędnej $x$.
\item $D$ -- średnica wewnętrzna diafragmy
\item $z_{\rm mid}$ położenie przecięcia powierzchni optycznej z osią optyczną
\item $z_{\rm cen}$ położenie środka krzywizny powierzchni optycznej
\end{itemize}

\section{Geometria pomiędzy sferami}

Przecięcie powierzchni prostopadłej do $\hat z$ w punkcie $z$ ze sferyczną powierzchnią.
\begin{equation}
\left[ 2R - ( z - z_{\rm mid} ) \right] \left( z - z_{\rm mid} \right) = x^2
\end{equation}

\section{Macierze ABCD}
W każdym położeniu $z$ wzdłuż osi optycznej można opisać promień światła za pomocą jego odległości $x$ od osi optycznej oraz kąta $\theta$, pod którym się  propaguje
\[
\vector{x}{\theta}.
\]
$x > 0$ oznacza, że promień znajduje się nad osią optyczną. $\theta>0$ oznacza, że promień podczas propagacji oddala się w górę od osi optycznej. 
Operacje na promieniu:
\begin{itemize}
 \item Propagacja o $d$ wzdłuż osi $z$ jest opisywana przez Macierze
 \[
  \ABCDmat{1}{d}{0}{1},
 \]
 tzn. 
 \[
  \ABCDmat{1}{d}{0}{1} \vector{x}{\theta} = \vector{x + \theta d}{\theta},
 \]
 gdzie kąt propagacji się nie zmienia, a odległość promienia od osi optycznej w ogólności się zmienia.
 
 \item Załamanie na sferze o promieniu $R$ jest opisane przez macierz
 \[
  \ABCDmat{1}{0}{\frac{n_1-n_2}{Rn_2}}{\frac{n_1}{n_2}},
 \]
 gdzie $n_1$ jest ośrodkiem, z którego wychodzi promień, a $n_2$ ośrodkiem, do którego się załamuje promień światła.

\end{itemize}



\end{document}
