\documentclass[11pt,a4paper,twocolumn]{article}
\usepackage[utf8]{inputenc}
\usepackage{polski}
\usepackage[polish]{babel}
\usepackage{tgtermes}
\usepackage{amsmath}
\usepackage{amsfonts}
%\usepackage{amssymb}
\usepackage{graphicx}
\usepackage[left=2cm,right=2cm,top=2cm,bottom=2cm]{geometry}
\author{Gronkiewicz, D., Pęcak, D.}
\title{Wzory do liczenia soczewek}
\begin{document}

\section{Notacja i konwencje}
Oś optyczna (oznaczona literą $z$) pozioma, propagacja z lewej do prawej.
\begin{itemize}
\item $R$ -- promień krzywizny granicy ośrodków. $R>0$ odpowiada wypukłości w lewą stronę.
\item $x$,$y$ -- współrzędne w płaszczyźnie optyki. W przypadku 1D używamy współrzędnej $x$.
\item $D$ -- średnica wewnętrzna diafragmy
\item $z_{\rm mid}$ położenie przecięcia powierzchni optycznej z osią optyczną
\item $z_{\rm cen}$ położenie środka krzywizny powierzchni optycznej
\end{itemize}

\section{Geometria pomiędzy sferami}

Przecięcie powierzchni prostopadłej do $\hat z$ w punkcie $z$ ze sferyczną powierzchnią.
\begin{equation}
\left[ 2R - ( z - z_{\rm mid} ) \right] \left( z - z_{\rm mid} \right) = x^2
\end{equation}

\end{document}